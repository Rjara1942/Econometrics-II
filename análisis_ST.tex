% Options for packages loaded elsewhere
\PassOptionsToPackage{unicode}{hyperref}
\PassOptionsToPackage{hyphens}{url}
%
\documentclass[
]{article}
\usepackage{amsmath,amssymb}
\usepackage{iftex}
\ifPDFTeX
  \usepackage[T1]{fontenc}
  \usepackage[utf8]{inputenc}
  \usepackage{textcomp} % provide euro and other symbols
\else % if luatex or xetex
  \usepackage{unicode-math} % this also loads fontspec
  \defaultfontfeatures{Scale=MatchLowercase}
  \defaultfontfeatures[\rmfamily]{Ligatures=TeX,Scale=1}
\fi
\usepackage{lmodern}
\ifPDFTeX\else
  % xetex/luatex font selection
\fi
% Use upquote if available, for straight quotes in verbatim environments
\IfFileExists{upquote.sty}{\usepackage{upquote}}{}
\IfFileExists{microtype.sty}{% use microtype if available
  \usepackage[]{microtype}
  \UseMicrotypeSet[protrusion]{basicmath} % disable protrusion for tt fonts
}{}
\makeatletter
\@ifundefined{KOMAClassName}{% if non-KOMA class
  \IfFileExists{parskip.sty}{%
    \usepackage{parskip}
  }{% else
    \setlength{\parindent}{0pt}
    \setlength{\parskip}{6pt plus 2pt minus 1pt}}
}{% if KOMA class
  \KOMAoptions{parskip=half}}
\makeatother
\usepackage{xcolor}
\usepackage[margin=1in]{geometry}
\usepackage{graphicx}
\makeatletter
\def\maxwidth{\ifdim\Gin@nat@width>\linewidth\linewidth\else\Gin@nat@width\fi}
\def\maxheight{\ifdim\Gin@nat@height>\textheight\textheight\else\Gin@nat@height\fi}
\makeatother
% Scale images if necessary, so that they will not overflow the page
% margins by default, and it is still possible to overwrite the defaults
% using explicit options in \includegraphics[width, height, ...]{}
\setkeys{Gin}{width=\maxwidth,height=\maxheight,keepaspectratio}
% Set default figure placement to htbp
\makeatletter
\def\fps@figure{htbp}
\makeatother
\setlength{\emergencystretch}{3em} % prevent overfull lines
\providecommand{\tightlist}{%
  \setlength{\itemsep}{0pt}\setlength{\parskip}{0pt}}
\setcounter{secnumdepth}{-\maxdimen} % remove section numbering
\ifLuaTeX
  \usepackage{selnolig}  % disable illegal ligatures
\fi
\usepackage{bookmark}
\IfFileExists{xurl.sty}{\usepackage{xurl}}{} % add URL line breaks if available
\urlstyle{same}
\hypersetup{
  pdftitle={Análisis de Series Temporales},
  pdfauthor={Ricardo Jara},
  hidelinks,
  pdfcreator={LaTeX via pandoc}}

\title{Análisis de Series Temporales}
\author{Ricardo Jara}
\date{2025-07-22}

\begin{document}
\maketitle

\section{Instalar si es necesario:}\label{instalar-si-es-necesario}

\section{install.packages(c(``forecast'', ``tseries'', ``readxl'',
``fpp2''))}\label{install.packagescforecast-tseries-readxl-fpp2}

library(forecast) library(tseries) library(readxl) library(fpp2) \#
incluye forecast, ggplot2, tseries, datasets, etc.

\section{Crear una Serie Temporal
Simulada}\label{crear-una-serie-temporal-simulada}

\section{Generamos 100 observaciones aleatorias normales con media 0 y
desviación estándar
1}\label{generamos-100-observaciones-aleatorias-normales-con-media-0-y-desviaciuxf3n-estuxe1ndar-1}

\section{Para asegurar
reproducibilidad}\label{para-asegurar-reproducibilidad}

set.seed(123) data \textless- ts(rnorm(100), frequency = 12) \#
Visualizamos la serie simulada

plot(data, main = ``Serie Temporal Aleatoria Simulada'', ylab =
``Valor'', xlab = ``Tiempo'') \# Test de Dickey-Fuller para
Estacionariedad \# ------------------------------------------

\section{Verifica si la serie tiene una raíz unitaria (es decir, si es
NO
estacionaria)}\label{verifica-si-la-serie-tiene-una-rauxedz-unitaria-es-decir-si-es-no-estacionaria}

adf\_test \textless- adf.test(data) print(adf\_test) \# Interpreta el
resultado: \# - p \textless{} 0.05: rechaza H0 → la serie es
estacionaria. \# - p \textgreater{} 0.05: no se rechaza H0 → la serie
podría ser no estacionaria.

\section{------------------------------------------}\label{section}

\section{Ajuste de Modelo AR(1)}\label{ajuste-de-modelo-ar1}

\section{------------------------------------------}\label{section-1}

\section{Ajustamos un modelo ARIMA(p=1, d=0, q=0) → modelo
AR(1)}\label{ajustamos-un-modelo-arimap1-d0-q0-modelo-ar1}

modelo\_ar \textless- arima(data, order = c(1, 0, 0))
summary(modelo\_ar) \# ------------------------------------------ \# 6.
Interpretación de Coeficientes AR \#
------------------------------------------

\section{Extraemos los coeficientes
estimados}\label{extraemos-los-coeficientes-estimados}

\section{p-valores}\label{p-valores}

\section{Extraemos los coeficientes
estimados}\label{extraemos-los-coeficientes-estimados-1}

coef\_ar \textless- modelo\_ar\(coef
# Calculamos errores estándar a partir de la matriz de varianza-covarianza
se_ar <- sqrt(diag(modelo_ar\)var.coef)) \# Calculamos valores z
(coeficiente / error estándar) z\_ar \textless- coef\_ar / se\_ar \#
Calculamos p-valores asociados p\_ar \textless- 2 * (1 -
pnorm(abs(z\_ar))) \# Consolidamos resultados en tabla resultados\_ar
\textless- data.frame(Coefficients = coef\_ar, Std\_Error = se\_ar,
Z\_value = z\_ar, P\_value = p\_ar) print(resultados\_ar) \# Si el
p-valor del coeficiente AR \textless{} 0.05, el efecto del rezago es
estadísticamente significativo. \#
------------------------------------------ \# Predicción con Modelo AR
\# ------------------------------------------

\section{Pronosticamos los próximos 10
periodos}\label{pronosticamos-los-pruxf3ximos-10-periodos}

pred\_ar \textless- predict(modelo\_ar, n.ahead = 10)

\section{Visualizamos la serie original y las
predicciones}\label{visualizamos-la-serie-original-y-las-predicciones}

ts.plot(data, pred\_ar\$pred, col = c(``black'', ``green''), lty = 1:2,
main = ``AR(1) - Predicción'', ylab = ``Valores'') \#
------------------------------------------ \# Ajuste de Modelo ARMA(1,1)
\# ------------------------------------------

\section{ARIMA(p=1, d=0, q=1): incorpora rezago de valores pasados (AR)
y errores pasados
(MA)}\label{arimap1-d0-q1-incorpora-rezago-de-valores-pasados-ar-y-errores-pasados-ma}

modelo\_arma \textless- arima(data, order = c(1, 0, 1))
summary(modelo\_arma) \# ------------------------------------------ \#
Predicción con Modelo ARMA \# ------------------------------------------

\section{Pronóstico a 10 pasos
adelante}\label{pronuxf3stico-a-10-pasos-adelante}

pred\_arma \textless- predict(modelo\_arma, n.ahead = 10) \#
Visualización opcional ts.plot(data, pred\_arma\$pred, col =
c(``black'', ``darkgreen''), lty = 1:2, main = ``ARMA(1,1) -
Predicción'', ylab = ``Valores'')

\section{CASO PRACTICO: Analizaremos la serie macroeconomica mensual
dolar observado (pesos por
dolar)}\label{caso-practico-analizaremos-la-serie-macroeconomica-mensual-dolar-observado-pesos-por-dolar}

\section{desde marzo de 1990 a junio de
2025}\label{desde-marzo-de-1990-a-junio-de-2025}

\#---------------------------------------------- \# Cargar archivo Excel
previamente importado como `dolar\_obs' \# dolar\_obs \textless-
read\_excel(``ruta/dolar.xlsx'') \# Reemplaza con tu ruta
\#--------------------------------------------

\section{renombramos las columnas}\label{renombramos-las-columnas}

colnames(dolar\_obs) \textless- c(``Fecha'', ``Valor'') \# cremos la
serie temporal, desde donde inicia e indicando su frecuencia mensual

tipo\_cambio \textless- ts(dolar\_obs\$Valor, start = c(1990, 3),
frequency = 12) \#Graficamos la serie original

plot(tipo\_cambio, main = ``Tipo de cambio nominal (CLP/USD)'', ylab =
``Pesos por dólar'', xlab = ``Año'')
\#---------------------------------------------\\
\# analizamos su ACF y aplicamos el Test-Dickey-Fuller acf(tipo\_cambio)
adf\_test\_tc \textless- adf.test(tipo\_cambio) print(adf\_test\_tc)
\#----------------------------------------------- \# cuando una serie es
de orden I(1) para volver una serie estacionaria de I(0) \# es necesario
calcular su primera diferencia y volver a testear \# la funcion ndiffs
nos indica las diferecencia que hay que calcular para que \# la serie se
vuelva estacionaria ndiffs(diff\_tc)
\#--------------------------------------------- \# calculamos la primera
diferencia diff\_tc \textless- diff(tipo\_cambio) \# Graficamos
plot(diff\_tc) \# obtenemos su funcion de autocorrelación acf(diff\_tc)

\section{probamos estacoionaridad con
Dickey-Fuller}\label{probamos-estacoionaridad-con-dickey-fuller}

adf\_test\_diff \textless- adf.test(diff\_tc) print(adf\_test\_diff)
\#-------------------------------------------------- \#análisis visual
de los graficos de autocorrelacion

par(mfrow = c(2, 2)) plot(tipo\_cambio, ylab = ``CLP/USD'')
acf(tipo\_cambio, main = ``No Estacionaria'') plot(diff\_tc, ylab =
``Cambio 1er orden'') acf(diff\_tc, main = ``Estacionaria'') par(mfrow =
c(1, 1))

\#--------------------------------------------------------------------
\# Ejemplo con la base uschange, contiene los cambios porcentuales
trimestrales de variables \# macroeconomicas de EE.UU \# variables:
Consumo, ingreso, produccion insdustrial, etc. \# para ello instalamos
el paquete fpp2 para trabajar con modelos ARIMA y \# analisis de series
temporales multivariadas
\#---------------------------------------------------

\section{cargamos la data y observamos su
estructura}\label{cargamos-la-data-y-observamos-su-estructura}

data(``uschange'') str(uschange)

\section{Generamos un grafico con la variable de
interés}\label{generamos-un-grafico-con-la-variable-de-interuxe9s}

autoplot(uschange{[}, ``Consumption''{]}) + ggtitle(``Cambio \%
trimestral en consumo (EE.UU.)'') + ylab(``\% cambio'') + xlab(``Año'')
\# ----------------------------------------------- \# como uschange esta
expresada en diferencias porcentuales, por lo tanto \# suele ser
estacionaria, comprebemos estacioonaridad (ACF + ADF)

par(mfrow = c(1, 2)) plot(uschange{[}, ``Consumption''{]}, main =
``Serie de consumo'') acf(uschange{[}, ``Consumption''{]}) par(mfrow =
c(1, 1))

\section{usamos Dickey-Fuller para confirmar
estacionaridad}\label{usamos-dickey-fuller-para-confirmar-estacionaridad}

adf.test(uschange{[}, ``Consumption''{]})

\#-----------------------------------------------------

\section{ya que confirmamos estacionaridad, podemos ajustar un modelo
ARIMA (p, d,
q)}\label{ya-que-confirmamos-estacionaridad-podemos-ajustar-un-modelo-arima-p-d-q}

modelo\_consumo \textless- auto.arima(uschange{[}, ``Consumption''{]})
summary(modelo\_consumo)
\#------------------------------------------------------ \# Validamos
supuestos del modelo \# buscamos que: \# ACF de los residuos no tengan
autocorrelacion \# Histograma se aproxime a normal \# p-value del test
Ljung-Box \textgreater{} 0,05 (residuos ≈ ruido blanco)

checkresiduals(modelo\_consumo)
\#-----------------------------------------

\section{Generar pronóstico para próximos 8 trimestres (2
años)}\label{generar-pronuxf3stico-para-pruxf3ximos-8-trimestres-2-auxf1os}

forecast\_consumo \textless- forecast(modelo\_consumo, h = 8)

\section{Visualizamos los resultados}\label{visualizamos-los-resultados}

autoplot(forecast\_consumo) + ggtitle(``Pronóstico de consumo
(EE.UU.)'') + ylab(``Cambio \% trimestral'')

install.packages(``tinytex'') tinytex::install\_tinytex()

\end{document}
